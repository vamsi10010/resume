

\documentclass[letterpaper,11pt]{article}

\usepackage{latexsym}
\usepackage[empty]{fullpage}
\usepackage{titlesec}
\usepackage{marvosym}
\usepackage[usenames,dvipsnames]{color}
\usepackage{verbatim}
\usepackage{enumitem}
\usepackage[hidelinks]{hyperref}
\usepackage{fancyhdr}
\usepackage[english]{babel}
\usepackage{tabularx}
\input{glyphtounicode}


%----------FONT OPTIONS----------
% sans-serif
% \usepackage[sfdefault]{FiraSans}
% \usepackage[sfdefault]{roboto}
% \usepackage[sfdefault]{noto-sans}
% \usepackage[default]{sourcesanspro}

% serif
% \usepackage{CormorantGaramond}
% \usepackage{charter}


\pagestyle{fancy}
\fancyhf{} % clear all header and footer fields
\fancyfoot{}
\renewcommand{\headrulewidth}{0pt}
\renewcommand{\footrulewidth}{0pt}

% Adjust margins
\addtolength{\oddsidemargin}{-0.5in}
\addtolength{\evensidemargin}{-0.5in}
\addtolength{\textwidth}{1in}
\addtolength{\topmargin}{-.5in}
\addtolength{\textheight}{1.0in}

\urlstyle{same}

\raggedbottom
\raggedright
\setlength{\tabcolsep}{0in}

% Sections formatting
\titleformat{\section}{
  \vspace{-4pt}\scshape\raggedright\large
}{}{0em}{}[\color{black}\titlerule \vspace{-5pt}]

% Ensure that generate pdf is machine readable/ATS parsable
\pdfgentounicode=1

%-------------------------
% Custom commands
\newcommand{\resumeItem}[1]{
  \item\small{
    {#1 \vspace{-2pt}}
  }
}

\newcommand{\resumeSubheading}[4]{
  \vspace{-2pt}\item
    \begin{tabular*}{0.97\textwidth}[t]{l@{\extracolsep{\fill}}r}
      \textbf{\small#1} & \small#2 \\
      \textit{\small#3} & \textit{\small #4} \\
    \end{tabular*}\vspace{-7pt}
}

\newcommand{\resumeSubSubheading}[2]{
    \item
    \begin{tabular*}{0.97\textwidth}{l@{\extracolsep{\fill}}r}
      \textit{\small#1} & \textit{\small #2} \\
    \end{tabular*}\vspace{-7pt}
}

\newcommand{\resumeProjectHeading}[2]{
    \item
    \begin{tabular*}{0.97\textwidth}{l@{\extracolsep{\fill}}r}
      \small#1 & \small#2 \\
    \end{tabular*}\vspace{-7pt}
}

\newcommand{\resumeSubItem}[1]{\resumeItem{#1}\vspace{-4pt}}

\renewcommand\labelitemii{$\vcenter{\hbox{\tiny$\bullet$}}$}

\newcommand{\resumeSubHeadingListStart}{\begin{itemize}[leftmargin=0.15in, label={}]}
\newcommand{\resumeSubHeadingListEnd}{\end{itemize}}
\newcommand{\resumeItemListStart}{\begin{itemize}}
\newcommand{\resumeItemListEnd}{\end{itemize}\vspace{-5pt}}

%-------------------------------------------
%%%%%%  RESUME STARTS HERE  %%%%%%%%%%%%%%%%%%%%%%%%%%%%


\begin{document}

%----------HEADING----------
% \begin{tabular*}{\textwidth}{l@{\extracolsep{\fill}}r}
%   \textbf{\href{http://sourabhbajaj.com/}{\Large Sourabh Bajaj}} & Email : \href{mailto:sourabh@sourabhbajaj.com}{sourabh@sourabhbajaj.com}\\
%   \href{http://sourabhbajaj.com/}{http://www.sourabhbajaj.com} & Mobile : +1-123-456-7890 \\
% \end{tabular*}

\begin{center}
    \textbf{\Huge \scshape Vamsi Deeduvanu} \\ \vspace{1pt}
    \small (765)-694-9091 $|$ \href{mailto:vamsi10010@gmail.com}{\underline{vamsi10010@gmail.com}} $|$ 
    \href{https://www.linkedin.com/in/vamsideeduvanu}{\underline{linkedin.com/in/vamsideeduvanu/}} $|$
    \href{https://github.com/vamsi10010}{\underline{github.com/vamsi10010}}
\end{center}

% Highly motivated student seeking Data Science and ML internships. Experienced in Java, Python, and C. Eager to learn more and expand technical expertise in the field of Machine Learning.


%-----------EDUCATION-----------
\section{Education}
  \resumeSubHeadingListStart
    \resumeSubheading
      {Purdue University}{Aug. 2022 -- May 2025 (exp.)}
      {Bachelor of Science in Computer Science and Data Science}{West Lafayette, IN}
      \resumeItemListStart
        \resumeItem{\textit{Coursework}: Object Oriented Programming, Multivariate Calculus, Linear Algebra, 
        \\Discrete Math, Probability, Statistics, Data Structures and Algorithms, Computer Architecture
        }
        \resumeItem{\textit{Clubs and Extracurriculars}: ML@Purdue, Purdue Astronomy Club, Intramural Soccer}
        \resumeItem{\textit{Honors}: Dean's List and Semester Honors (Fall 2022, Spring 2023)}
        \resumeItem{\textit{GPA}: 4.0/4.0}
      \resumeItemListEnd
    % \resumeSubheading
    %   {Resonance Junior College}{June 2020 -- May 2022}
    %   {Intermediate}{Hyderabad, India}
    %   \resumeItemListStart
    %   \resumeItem{\textit{JEE Mains}: 99.49th percentile all India}
    %   % \resumeItem{\textit{Intermediate Public Examinations}: 98.2\%}
    % \resumeItemListEnd
  \resumeSubHeadingListEnd


%-----------EXPERIENCE-----------
\section{Experience}
  \resumeSubHeadingListStart
  \resumeSubheading{Undergraduate Researcher}{Aug. 2023 -- Present}
  {Purdue Vertically Integrated Projects}{West Lafayette, IN}
    \resumeItemListStart
      \resumeItem{Collaborated with local industry partners to establish an IIoT-based smart machine monitoring framework to improve manufacturing efficiency.}
      \resumeItem{Developed a data pipeline to collect and process real-time machine data from MTConnect agents.}
      \resumeItem{Labelled and annotated sensor data to train a deep learning model to predict chatter and other failures.}
      \resumeItem{Awarded 2nd Best Poster for presentation at 2023 Fall Undergraduate Research Expo.}
    \resumeItemListEnd

  \resumeSubheading{Undergraduate Teaching Assistant}{Aug. 2023 -- Present}
      {Department of Computer Science, Purdue University}{West Lafayette, IN}
      \resumeItemListStart
        \resumeItem{Provided instructional assistance as TA to students in CS 24000 (Programming in C) and CS 19300 (Tools).}
        \resumeItem{Enhanced student's learning outcomes by conducting weekly lab sessions and office hours for 40+ students.}
        % \resumeItem{Grade assignments and exams to provide feedback to students and help them improve.}
        \resumeItem{Actively monitored online discussion forums to resolve student's questions outside of class.}
      \resumeItemListEnd

    \resumeSubheading
      {Data Science Researcher}{Aug. 2022 -- May 2023}
      {Battelle}{West Lafayette, IN}
      \resumeItemListStart
      \resumeItem{Researched hyperband and population-based training algorithms to tune hyperparameters of NLP models.}
        % \resumeItem{Partnered with Battelle on researching hyper-parameter optimization procedures for NLP models resulting in significant improvement in performance. }
        \resumeItem{Fine-tuned an LLM from HuggingFace to accurately identify adverse drug events in electronic health records.}
        % \resumeItem{Applied hyperband and population-based training algorithms from RayTune to tune hyperparameters and improved overall f1 score by more than 20\%.}
        \resumeItem{Boosted overall f1 score by more than 20\% using hyperparameter tuning algorithms from RayTune.}
        % \resumeItem{Delivered updates in weekly sprint meetings with client and documented a standard operating procedure as a reference for future projects.}
        \resumeItem{Established a successful standard operating procedure for hyperparameter tuning as reference for future projects.}
        \resumeItem{Presented research poster at The Data Mine Symposium and demonstrated entity recognition on a live document.}
      \resumeItemListEnd
      
% -----------Multiple Positions Heading-----------
%    \resumeSubSubheading
%     {Software Engineer I}{Oct 2014 - Sep 2016}
%     \resumeItemListStart
%        \resumeItem{Apache Beam}
%          {Apache Beam is a unified model for defining both batch and streaming data-parallel processing pipelines}
%     \resumeItemListEnd
%    \resumeSubHeadingListEnd
%-------------------------------------------

  \resumeSubHeadingListEnd


%-----------PROJECTS-----------
\section{Projects}
    \resumeSubHeadingListStart
      \resumeProjectHeading
          {\textbf{cgrad} $|$ \emph{C, cmocka, Deep Learning}}{Aug. 2023 -- Present}
          \resumeItemListStart
            \resumeItem{Programmed a lightweight backpropagation and neural network library for deep learning in C.}
            % \resumeItem{Implemented a directed acyclic graph similar to PyTorch to represent computational graph of a neural network and perform backpropagation to calculate gradients.}
            % \resumeItem{Designed an artificial neural network library that supports multiple layers, activation functions, gradient descent methods, and regularization options to enable model flexibility.}
            \resumeItem{Optimized memory usage during model training by managing heap space with dynamic memory allocation.}
            % \resumeItem{Applied cgrad to train an ANN to classify handwritten digits from MNIST dataset with 96\% accuracy, using less than 2GB of memory during training.}
            \resumeItem{Successfully classified handwritten digits with 96\% accuracy, using less than 2GB of memory during training.}
            % \resumeItem{Created unit tests for library using cmocka framework to automate testing process and ensure functionality.}
            \resumeItem{Automated testing process and ensured functionality by creating unit tests using cmocka framework.}
          \resumeItemListEnd

      \resumeProjectHeading
          {\textbf{YourCollege} $|$ \emph{Python, Scikit-Learn, Pandas, Streamlit}}{Jan. 2023 -- Present}
          \resumeItemListStart
            \resumeItem{Developed a college recommender application to assist high school students in college search.}
            \resumeItem{Collected data from multiple sources and performed data cleaning and feature engineering for classification.}
            \resumeItem{Trained an unsupervised learning model to classify colleges tailored to every user's preferences.}
            \resumeItem{Built and deployed application on web through Streamlit to make it accessible to users.}
          \resumeItemListEnd
      \resumeProjectHeading
          {\textbf{Time Series Forecasting} $|$ \emph{Python, Statsmodels, Pandas, Keras, Streamlit}}{Sep. 2022 -- May 2023}          
          \resumeItemListStart
            % \resumeItem{Collaborated with members of ML@Purdue to forecast PM-10 pollution levels in California.}
            \resumeItem{Created a dashboard to accurately predict air pollution levels using time series forecasting techniques.}
            % \resumeItem{Implemented an ARIMA model from statsmodels library and an LSTM model from Keras to predict PM-10 levels achieving high accuracy rates and providing valuable insights.}
            \resumeItem{Achieved high accuracy rates by implementing ARIMA and LSTM models to predict PM-10 levels.}
            \resumeItem{Designed an interactive dashboard to visualize predictions and provide valuable insights to users.}
          \resumeItemListEnd
    \resumeSubHeadingListEnd



%
%-----------PROGRAMMING SKILLS-----------
\section{Skills}
 \begin{itemize}[leftmargin=0.15in, label={}]
    \small{\item{
     \textbf{Languages}{: Python, C, C++, Java, SQL (Postgres), R, LaTeX, x86-64 Assembly} \\
     \textbf{Developer Tools}{: Git, Bash, Linux, VS Code, Jupyter, JetBrains, MTConnect, Agile Methodologies, XML} \\
     \textbf{Libraries}{: PyTorch, HuggingFace, Keras, RayTune, Scikit-Learn, Streamlit, Pandas, NumPy, Matplotlib} \\
    %  \textbf{Operating Systems}{: Linux, Windows} \\ 
    }}
 \end{itemize}


%-------------------------------------------
\end{document}
