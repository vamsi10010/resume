

\documentclass[letterpaper,11pt]{article}

\usepackage{latexsym}
\usepackage[empty]{fullpage}
\usepackage{titlesec}
\usepackage{marvosym}
\usepackage[usenames,dvipsnames]{color}
\usepackage{verbatim}
\usepackage{enumitem}
\usepackage[hidelinks]{hyperref}
\usepackage{fancyhdr}
\usepackage[english]{babel}
\usepackage{tabularx}
\input{glyphtounicode}


%----------FONT OPTIONS----------
% sans-serif
% \usepackage[sfdefault]{FiraSans}
% \usepackage[sfdefault]{roboto}
% \usepackage[sfdefault]{noto-sans}
% \usepackage[default]{sourcesanspro}

% serif
% \usepackage{CormorantGaramond}
% \usepackage{charter}


\pagestyle{fancy}
\fancyhf{} % clear all header and footer fields
\fancyfoot{}
\renewcommand{\headrulewidth}{0pt}
\renewcommand{\footrulewidth}{0pt}

% Adjust margins
\addtolength{\oddsidemargin}{-0.5in}
\addtolength{\evensidemargin}{-0.5in}
\addtolength{\textwidth}{1in}
\addtolength{\topmargin}{-.5in}
\addtolength{\textheight}{1.0in}

\urlstyle{same}

\raggedbottom
\raggedright
\setlength{\tabcolsep}{0in}

% Sections formatting
\titleformat{\section}{
  \vspace{-6pt}\scshape\raggedright\large
}{}{0em}{}[\color{black}\titlerule \vspace{-5pt}]

% Ensure that generate pdf is machine readable/ATS parsable
\pdfgentounicode=1

%-------------------------
% Custom commands
\newcommand{\resumeItem}[1]{
  \item\small{
    {#1 \vspace{-2pt}}
  }
}

\newcommand{\resumeSubheading}[4]{
  \vspace{-2pt}\item
    \begin{tabular*}{0.97\textwidth}[t]{l@{\extracolsep{\fill}}r}
      \textbf{\small#1} & \small#2 \\
      \textit{\small#3} & \textit{\small #4} \\
    \end{tabular*}\vspace{-7pt}
}

\newcommand{\resumeSubSubheading}[2]{
    \item
    \begin{tabular*}{0.97\textwidth}{l@{\extracolsep{\fill}}r}
      \textit{\small#1} & \textit{\small #2} \\
    \end{tabular*}\vspace{-7pt}
}

\newcommand{\resumeProjectHeading}[2]{
    \item
    \begin{tabular*}{0.97\textwidth}{l@{\extracolsep{\fill}}r}
      \small#1 & \small#2 \\
    \end{tabular*}\vspace{-7pt}
}

\newcommand{\resumeSubItem}[1]{\resumeItem{#1}\vspace{-4pt}}

\renewcommand\labelitemii{$\vcenter{\hbox{\tiny$\bullet$}}$}

\newcommand{\resumeSubHeadingListStart}{\begin{itemize}[leftmargin=0.15in, label={}]}
\newcommand{\resumeSubHeadingListEnd}{\end{itemize}}
\newcommand{\resumeItemListStart}{\begin{itemize}}
\newcommand{\resumeItemListEnd}{\end{itemize}\vspace{-6pt}}

%-------------------------------------------
%%%%%%  RESUME STARTS HERE  %%%%%%%%%%%%%%%%%%%%%%%%%%%%


\begin{document}

%----------HEADING----------
\begin{center}
    \textbf{\Huge \scshape Vamsi Deeduvanu} \\ \vspace{1pt}
    \small (765)-694-9091 $|$ \href{mailto:vamsi10010@gmail.com}{\underline{vamsi10010@gmail.com}} $|$ 
    \href{https://www.linkedin.com/in/vamsideeduvanu}{\underline{linkedin.com/in/vamsideeduvanu/}} $|$
    \href{https://github.com/vamsi10010}{\underline{github.com/vamsi10010}}
\end{center}

% Highly motivated student seeking Data Science and ML internships. Experienced in Java, Python, and C. Eager to learn more and expand technical expertise in the field of Machine Learning.

\vspace{-14pt}
%-----------EDUCATION-----------
\section{Education}
  \resumeSubHeadingListStart
    \resumeSubheading
      {Purdue University}{Aug. 2024 -- May 2026}
      {Masters of Science in Computer Science}{\textit{GPA}: 4.0/4.0}
    \resumeSubheading
      {Purdue University}{Aug. 2022 -- May 2025}
      {Bachelor of Science in Computer Science}{\textit{GPA}: 3.99/4.0}
      \resumeItemListStart
        \resumeItem{\textit{Coursework}: OOP, DSA, Computer Architecture, Systems Programming, Compilers, AI, \\Machine Learning, NLP, Robotics, ML Systems, Randomized Algorithms, Statistical Theory
        }
        \resumeItem{\textit{Honors}: Dean's List and Semester Honors (6x), L3Harris Scholarship, UG Research Expo Award}
      \resumeItemListEnd
  \resumeSubHeadingListEnd


%-----------EXPERIENCE-----------
\section{Experience}
  \resumeSubHeadingListStart
  \resumeSubheading{Software Development Engineer Intern}{May 2025 -- Aug. 2025}
      {Amazon, Project Kuiper}{Redmond, WA}
      \resumeItemListStart
        \resumeItem{Developed a cost tracing service with RESTful API improving supply chain cost visibility for Project Kuiper.}
        \resumeItem{Deployed a scalable serverless data pipeline on AWS achieving sub-10ms queries on millions of cost events.}
        \resumeItem{Built an MCP agent using Neo4j and Claude enabling non-tech stakeholders to perform natural language queries.}
      \resumeItemListEnd
  \resumeSubheading{AI/ML Intern}{May 2024 -- Aug. 2024}
      {Volvo Group}{Hagerstown, MD}
      \resumeItemListStart
        \resumeItem{Designed an edge AI pipeline to identify service bottlenecks on factory floor using YOLOv8n and PaddleOCR.}
        \resumeItem{Developed a live web interface using Streamlit to monitor KPIs such as truck count and takt time on-site.}
        % \resumeItem{Utilized Azure ML Studio conduct labeling, automate model training, and generate containers for deployment.}
        % Leveraged Thingworx API to store and visualize truck service information in real-time.}
        % \resumeItem{Integrated data from turbo-compound engine manufacturing lines to Thingworx platform to identify bottlenecks, improve part traceability,
        % and display visualizations on factory floor in real-time.}
        \resumeItem{Leveraged VAR and LSTM models to forecast service requests to reduce downtimes and improve service efficiency.}
      \resumeItemListEnd
  \resumeSubheading{Undergraduate Teaching Assistant}{Aug. 2023 -- May 2025}
      {Department of Computer Science, Purdue University}{West Lafayette, IN}
      \resumeItemListStart
        % \resumeItem{Provided instructional assistance to students in CS 240 (Programming in C) and CS 252 (Systems Programming).}
        \resumeItem{Mentored undergraduate students on foundational concepts in C programming and systems programming courses.}
        \resumeItem{Developed programming assignments and test frameworks to automate evaluation of students' understanding.}
        \resumeItem{Led weekly lab sessions and office hours for 40+ students enhancing student learning outcomes and grades.}
        % \resumeItem{Grade assignments and exams to provide feedback to students and help them improve.}
        % \resumeItem{Actively monitored online discussion forums to resolve student's questions outside of class.}
      \resumeItemListEnd
  \resumeSubheading{Undergraduate Researcher}{Aug. 2023 -- Dec. 2023}
  {TinyML/IIoT, Purdue University}{West Lafayette, IN}
    \resumeItemListStart
      % \resumeItem{Collaborated with local industry partners to establish a low-cost IIoT-based machine monitoring framework.}
      \resumeItem{Developed a data pipeline to collect and process real-time CNC machine and sensor data using MTConnect.}
      \resumeItem{Labelled and annotated sensor data to train a CNN to predict machining failures with \(> 90\%\) accuracy.}
      % \resumeItem{Awarded 2nd Best Poster for presentation at 2023 Fall Undergraduate Research Expo.}
    \resumeItemListEnd

    \resumeSubheading
      {Data Science Researcher}{Aug. 2022 -- May 2023}
      {Battelle}{West Lafayette, IN}
      \resumeItemListStart
      \resumeItem{Conducted research on hyperparameter tuning algorithms for LLMs and established an SOP for future projects.}
        % \resumeItem{Partnered with Battelle on researching hyper-parameter optimization procedures for NLP models resulting in significant improvement in performance. }
        \resumeItem{Fine-tuned BioBERT from HuggingFace to accurately identify adverse drug events in electronic health records.}
        % \resumeItem{Applied hyperband and population-based training algorithms from RayTune to tune hyperparameters and improved overall f1 score by more than 20\%.}
        \resumeItem{Boosted overall f1 score by more than 20\% using hyperband and population-based algorithms from RayTune.}
        % \resumeItem{Delivered updates in weekly sprint meetings with client and documented a standard operating procedure as a reference for future projects.}
        % \resumeItem{Established a successful standard operating procedure for hyperparameter tuning as reference for future projects.}
        % \resumeItem{Presented research poster at The Data Mine Symposium and demonstrated entity recognition on a live document.}
      \resumeItemListEnd
    \resumeSubHeadingListEnd
      
% -----------Multiple Positions Heading-----------
%    \resumeSubSubheading
%     {Software Engineer I}{Oct 2014 - Sep 2016}
%     \resumeItemListStart
%        \resumeItem{Apache Beam}
%          {Apache Beam is a unified model for defining both batch and streaming data-parallel processing pipelines}
%     \resumeItemListEnd
%    \resumeSubHeadingListEnd
%-------------------------------------------



%-----------PROJECTS-----------
\section{Projects}
    \resumeSubHeadingListStart
    % \resumeProjectHeading
    %       {\textbf{DuetDanceMotion} $|$ \emph{Python, PyTorch, SMPLX, Blender}}{May. 2024 - Present}
    %       \resumeItemListStart
    %         \resumeItem{Investigated generative models for synthesizing realistic human dance motion using text prompts and music cues.}
    %         \resumeItem{Collected more than 6 hours of motion capture data of professional dancers to train generative models.}
    %         \resumeItem{Developed a pipeline to convert mocap data to SMPLX format and visualize using Blender.}
    %       \resumeItemListEnd
      \resumeProjectHeading
        {\textbf{UNIX Shell} $|$ \emph{C, C++, Flex, Bison, UNIX}}{Jan. 2024 - May. 2024}
        \resumeItemListStart
          \resumeItem{Built a UNIX shell interpreter with support for complex command parsing and subshell execution.}
          \resumeItem{Integrated wildcard expansion using C++ regex to execute commands on multiple files simultaneously.}
          \resumeItem{Designed a feature rich line editor supporting command history, path completion, and prompt customization.}
        \resumeItemListEnd
      \resumeProjectHeading
          {\textbf{hirehack} $|$ \emph{Python, JavaScript, PyTorch, HuggingFace, PRAAT, WebSpeech API}}{Jan. 2024 - Feb. 2024}
          \resumeItemListStart
            \resumeItem{Developed a Chrome extension to automatically analyze interview performance and provide feedback.}
            \resumeItem{Integrated facial emotion, prosodic, and lexical features into a multi-modal model to score interview performance.}
            % \resumeItem{Designed modules to analyze prosodic and facial emotion features using PRAAT and Keras.}
            % \resumeItem{Automated recording and transcription of interviewee responses using WebSpeech API and performed sentiment analysis to extact lexical features.}
            % \resumeItem{Trained an FNN using PyTorch to predict interviewee performance based on extracted features.}
            \resumeItem{Interfaced Mixtral-7B from HuggingFace API to interpret model output and generate feedback in real-time.}
          \resumeItemListEnd
      \resumeProjectHeading
          {\textbf{cgrad} $|$ \emph{C, cmocka, Deep Learning}}{Aug. 2023 -- Sep. 2023}
          \resumeItemListStart
            \resumeItem{Created a lightweight neural network library from scratch in C achieving 96\% accuracy on MNIST dataset.}
            % \resumeItem{Implemented a directed acyclic graph similar to PyTorch to represent computational graph of a neural network and perform backpropagation to calculate gradients.}
            \resumeItem{Programmed support for layers, activation functions, gradient descent methods, and regularization options.}
            % \resumeItem{Optimized memory usage during model training by managing heap space with dynamic memory allocation.}
            % \resumeItem{Applied cgrad to train an ANN to classify handwritten digits from MNIST dataset with 96\% accuracy, using less than 2GB of memory during training.}
            % \resumeItem{Successfully classified handwritten digits with 96\% accuracy, using less than 2GB of memory during training.}
            % \resumeItem{Created unit tests for library using cmocka framework to automate testing process and ensure functionality.}
            \resumeItem{Automated testing process and ensured functionality by creating unit tests using cmocka framework.}
          \resumeItemListEnd

      % \resumeProjectHeading
      %     {\textbf{YourCollege} $|$ \emph{Python, Scikit-Learn, Pandas, Streamlit}}{Jan. 2023 -- Apr. 2023}
      %     \resumeItemListStart
      %       \resumeItem{Developed a college recommender application to assist high school students in college search.}
      %       \resumeItem{Collected data from multiple sources and performed data cleaning and feature engineering for classification.}
      %       \resumeItem{Trained an unsupervised learning model to classify colleges tailored to every user's preferences.}
      %       \resumeItem{Built and deployed application on web through Streamlit to make it accessible to users.}
      %     \resumeItemListEnd
      % \resumeProjectHeading
      %     {\textbf{Time Series Forecasting} $|$ \emph{Python, Statsmodels, Pandas, Keras, Streamlit}}{Sep. 2022 -- May 2023}          
      %     \resumeItemListStart
      %       % \resumeItem{Collaborated with members of ML@Purdue to forecast PM-10 pollution levels in California.}
      %       \resumeItem{Created a dashboard to accurately predict air pollution levels using time series forecasting techniques.}
      %       % \resumeItem{Implemented an ARIMA model from statsmodels library and an LSTM model from Keras to predict PM-10 levels achieving high accuracy rates and providing valuable insights.}
      %       \resumeItem{Achieved high accuracy rates by implementing ARIMA and LSTM models to predict PM-10 levels.}
      %       \resumeItem{Designed an interactive dashboard to visualize predictions and provide valuable insights to users.}
      %     \resumeItemListEnd
    \resumeSubHeadingListEnd



%
%-----------PROGRAMMING SKILLS-----------
\section{Skills}
 \begin{itemize}[leftmargin=0.15in, label={}]
    \small{\item{
     \textbf{Languages}{: Python, C, C++, Java, SQL, R, JavaScript, TypeScript, Smithy, LaTeX, x86-64 Assembly} \\
     \textbf{Developer Tools}{: Git, Bash, Linux, MacOS, LLVM, AWS, IaC, Docker, Neo4j, REST API, uv} \\
     \textbf{Libraries}{: PyTorch, HuggingFace, Keras, Ultralytics, RayTune, Streamlit, PySpark, Tensorflow, AWS CDK} \\
    %  \textbf{Operating Systems}{: Linux, Windows} \\ 
    }}
 \end{itemize}


%-------------------------------------------
\end{document}
